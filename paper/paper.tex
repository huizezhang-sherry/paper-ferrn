% !TeX program = pdfLaTeX
\documentclass[12pt]{article}
\usepackage{amsmath}
\usepackage{graphicx,psfrag,epsf}
\usepackage{enumerate}
\usepackage{natbib}
\usepackage{textcomp}
\usepackage[hyphens]{url} % not crucial - just used below for the URL
\usepackage{hyperref}
\providecommand{\tightlist}{%
  \setlength{\itemsep}{0pt}\setlength{\parskip}{0pt}}

%\pdfminorversion=4
% NOTE: To produce blinded version, replace "0" with "1" below.
\newcommand{\blind}{0}

% DON'T change margins - should be 1 inch all around.
\addtolength{\oddsidemargin}{-.5in}%
\addtolength{\evensidemargin}{-.5in}%
\addtolength{\textwidth}{1in}%
\addtolength{\textheight}{1.3in}%
\addtolength{\topmargin}{-.8in}%

%% load any required packages here


\usepackage{color}
\usepackage{fancyvrb}
\newcommand{\VerbBar}{|}
\newcommand{\VERB}{\Verb[commandchars=\\\{\}]}
\DefineVerbatimEnvironment{Highlighting}{Verbatim}{commandchars=\\\{\}}
% Add ',fontsize=\small' for more characters per line
\usepackage{framed}
\definecolor{shadecolor}{RGB}{248,248,248}
\newenvironment{Shaded}{\begin{snugshade}}{\end{snugshade}}
\newcommand{\AlertTok}[1]{\textcolor[rgb]{0.94,0.16,0.16}{#1}}
\newcommand{\AnnotationTok}[1]{\textcolor[rgb]{0.56,0.35,0.01}{\textbf{\textit{#1}}}}
\newcommand{\AttributeTok}[1]{\textcolor[rgb]{0.77,0.63,0.00}{#1}}
\newcommand{\BaseNTok}[1]{\textcolor[rgb]{0.00,0.00,0.81}{#1}}
\newcommand{\BuiltInTok}[1]{#1}
\newcommand{\CharTok}[1]{\textcolor[rgb]{0.31,0.60,0.02}{#1}}
\newcommand{\CommentTok}[1]{\textcolor[rgb]{0.56,0.35,0.01}{\textit{#1}}}
\newcommand{\CommentVarTok}[1]{\textcolor[rgb]{0.56,0.35,0.01}{\textbf{\textit{#1}}}}
\newcommand{\ConstantTok}[1]{\textcolor[rgb]{0.00,0.00,0.00}{#1}}
\newcommand{\ControlFlowTok}[1]{\textcolor[rgb]{0.13,0.29,0.53}{\textbf{#1}}}
\newcommand{\DataTypeTok}[1]{\textcolor[rgb]{0.13,0.29,0.53}{#1}}
\newcommand{\DecValTok}[1]{\textcolor[rgb]{0.00,0.00,0.81}{#1}}
\newcommand{\DocumentationTok}[1]{\textcolor[rgb]{0.56,0.35,0.01}{\textbf{\textit{#1}}}}
\newcommand{\ErrorTok}[1]{\textcolor[rgb]{0.64,0.00,0.00}{\textbf{#1}}}
\newcommand{\ExtensionTok}[1]{#1}
\newcommand{\FloatTok}[1]{\textcolor[rgb]{0.00,0.00,0.81}{#1}}
\newcommand{\FunctionTok}[1]{\textcolor[rgb]{0.00,0.00,0.00}{#1}}
\newcommand{\ImportTok}[1]{#1}
\newcommand{\InformationTok}[1]{\textcolor[rgb]{0.56,0.35,0.01}{\textbf{\textit{#1}}}}
\newcommand{\KeywordTok}[1]{\textcolor[rgb]{0.13,0.29,0.53}{\textbf{#1}}}
\newcommand{\NormalTok}[1]{#1}
\newcommand{\OperatorTok}[1]{\textcolor[rgb]{0.81,0.36,0.00}{\textbf{#1}}}
\newcommand{\OtherTok}[1]{\textcolor[rgb]{0.56,0.35,0.01}{#1}}
\newcommand{\PreprocessorTok}[1]{\textcolor[rgb]{0.56,0.35,0.01}{\textit{#1}}}
\newcommand{\RegionMarkerTok}[1]{#1}
\newcommand{\SpecialCharTok}[1]{\textcolor[rgb]{0.00,0.00,0.00}{#1}}
\newcommand{\SpecialStringTok}[1]{\textcolor[rgb]{0.31,0.60,0.02}{#1}}
\newcommand{\StringTok}[1]{\textcolor[rgb]{0.31,0.60,0.02}{#1}}
\newcommand{\VariableTok}[1]{\textcolor[rgb]{0.00,0.00,0.00}{#1}}
\newcommand{\VerbatimStringTok}[1]{\textcolor[rgb]{0.31,0.60,0.02}{#1}}
\newcommand{\WarningTok}[1]{\textcolor[rgb]{0.56,0.35,0.01}{\textbf{\textit{#1}}}}



\begin{document}


\def\spacingset#1{\renewcommand{\baselinestretch}%
{#1}\small\normalsize} \spacingset{1}


%%%%%%%%%%%%%%%%%%%%%%%%%%%%%%%%%%%%%%%%%%%%%%%%%%%%%%%%%%%%%%%%%%%%%%%%%%%%%%

\if0\blind
{
  \title{\bf Title here}

  \author{
        Author 1 \thanks{The authors gratefully acknowledge \ldots{}} \\
    Department of YYY, University of XXX\\
     and \\     Author 2 \\
    Department of ZZZ, University of WWW\\
      }
  \maketitle
} \fi

\if1\blind
{
  \bigskip
  \bigskip
  \bigskip
  \begin{center}
    {\LARGE\bf Title here}
  \end{center}
  \medskip
} \fi

\bigskip
\begin{abstract}
\begin{enumerate}
\def\labelenumi{\arabic{enumi}.}
\tightlist
\item
  check consistency of using PP for projection pursuit and PPI for
  projection pursuit index
\end{enumerate}
\end{abstract}

\noindent%
{\it Keywords:} 3 to 6 keywords, that do not appear in the title
\vfill

\newpage
\spacingset{1.45} % DON'T change the spacing!

\hypertarget{introduction}{%
\section{Introduction}\label{introduction}}

\hypertarget{tour}{%
\subsection{Tour}\label{tour}}

Tour provides a way to explore multivariate data interactively via an
established tour path. A tour path is formed by interpolating between
randomly generated plane. Different types of tours are available depends
on the purpose of the exploration i.e.~a grand tour is suitable for
randomly exploring the data from different angles; a guided tour detects
a particular structure in the data. a manual tour allows user to
manually control the projection \citep{cook1997manual}.

\hypertarget{guided-tour}{%
\subsection{Guided tour}\label{guided-tour}}

Guided tour is usually used in conjunction projection pursuit, a method
coined by \citet{friedman1974projection} to detect ``interesting''
low-diemsion projection of multivariate data. A projection pursuit
requires the definition of a projection pursuit index function and an
optimisation routine. The projection pursuit index measures the
``interestingness'' of data defined as its departure from normality.
Numerous indices have been proposed in the literature, including
lengendre index \citep{friedman1974projection}, hermite index
\citep{hall1989polynomial}, natural hermite index
\citep{cook1993projection}, chi-square index
\citep{posse1995projection}, LDA index \citep{lee2005projection} and PDA
index \citep{lee2010projection}. An optimisation routine is required to
find the projection basis (thus projection) that maximises the
projection pursuit index. The discussion of existing optimisation
procedure will be discussed in the next section.

Guided tour creates visualisation for the projections found by
projection pursuit by constructing a tour path. An illustration modified
from \citep{buja2005computational} entails the geodesic interpolation
bewteen the plane generated by projection pursuit.

\includegraphics[width=7.75in]{/Users/hzha400/Documents/3.PhD/paper-tour-vis/figures/tour_path_new}

\hypertarget{optimisation-methods-in-projection-pursuit-literature}{%
\subsection{Optimisation methods in projection pursuit
literature}\label{optimisation-methods-in-projection-pursuit-literature}}

As \citet{friedman1974projection} said ``\ldots{}, the technique use for
maximizing the (one- and two-dimensional) projection index strongly
influences both the statistical and the com- putational aspects of the
procedure.'' The quality of the optimisation procedure largely affect
the tour view and thus, the interesting projection one could possibly
observe. An ideal optimisation procedure needs to have following
characteristics:

\begin{itemize}
\item
  \emph{Being able to handle non-differentiable index function}: the
  index function could be noisy and non-differentible
\item
  \emph{Being able to optimise with constraints}: the projection matrix
  is restricted to an orthornormal matrix.
\item
  \emph{Being able to reveal both local and global maximum}: Although
  the primary interest is to find the global maximum, distinct local
  structures are also of our interest.
\end{itemize}

Below present three existing methods in tour.

\citet{posse1995projection} presented a random search algorithm that
samples new basis in the neighbourhood of the current basis. The
neighbourhood is defined via the radius of the p-dimenionsal sphere,
\(c\). The new basis is taken as the target basis if it has higher index
value, or the sampling continues. If no basis is found to have higher
index value after a certain humber of tries \(n\), the radius \(c\) is
halved. The algorithm stops when the maximum numbere of iteration is
attained or the radius \(c\) is less than a pre-determined number.
{[}Pursuit package uses this method and it works greate! But I don't
think we implement this - although don't think it is too hard to do
it{]}.

\citet{cook1995grand} explained the use of a gradient ascent
optimisation with the assumption that the index function is continuous
and differentiable. Since some indices could be non-differentiable, the
computation of derivative is replaced by a psudo-derivative of
evaluating five randomly generated directions in a tiny nearby
neighbourhood. Taking a step on the straight derivative direction has
been modified to maximise the projection pursuit index along the
geodesic direction.

Simulated annealing
\citep[\citet{kirkpatrick1983optimization}]{bertsimas1993simulated} is a
non-derivative procedure based on a non-increasing cooling scheme
\(T(i)\). Given an initial \(T_0\), the temperature at iteration \(i\)
is defined as \(T(i) = \frac{T_0}{log(i + 1)}\). The simulated annealing
algorithm works as follows. Given a neighbourhood parameter \(\alpha\)
and a randomly generated orthonormal basis \(B\), a candidate basis is
constructed as \(B_j = (1- \alpha)B_i + \alpha B\) where \(B_i\) is the
current basis. If the index value of the candidate basis is larger than
the one of the current basis, the candidate basis becomes the target
basis. If it is smaller, the candidate is accepted with probability
\(A = \min \left(\exp(-\frac{I(B_j) - I(B_i)}{T(i)}), 1 \right)\) where
\(I(.)\) is the index function.

\hypertarget{problems-and-difficulties-in-pp-optimisation}{%
\subsection{problems and difficulties in PP
optimisation}\label{problems-and-difficulties-in-pp-optimisation}}

Below listed several issues in projection pursuit optimisation. Some of
them are general problems in optimisation literature, while others are
more specific for PP optimisation.

\begin{itemize}
\item
  \emph{Finding global maximum}: Although finding local maximum is
  relatively easy with developed algorithms, it is generally hard to
  guarantee global maximum in a problem where the objective function is
  complex or the number of decision variables is large. Also, there are
  discussions on how to avoid getting trapped in a local optimal in the
  literature.
\item
  \emph{optimising on non-smooth function}: When the objective function
  is non-differentiable, derivative information can not be obtained.
  This means traditional gradient- or Hessian- based methods are not
  feasible, stochastic optimisation method could be an alternative to
  solve these problems.
\item
  \emph{computation speed}: The optimisation procedure needs to be fast
  to compute because tours produces real-time animation of the projected
  data.
\item
  \emph{consistency result in stochastic optimisation}: In stochastic
  algorithm, researchers usually set a seed to ensure the algorithm
  producing the same result for every run. While this practice supports
  reproducibility, less efforts has been made to guarantee different
  seeds will provide the same result.
\item
  \emph{high-dimensional decision variable}: In projection pursuit, the
  decision variable is the entry in the projection matrix, which is
  usually high-dimensional. Researcher would be better off if they can
  understand the relative position of different projection matrix in the
  high-dimensional space.
\item
  \emph{role of interpolation in PP optimisation}: An optimisation
  procedure usually involves iteratively finding projection bases that
  maximise the index function, while tour requires geodesic
  interpolation between these bases to produce a continuous view for the
  users. It would be interesting to see if the interpolated bases could,
  in reverse, help the optimisation reach faster convergence.
\end{itemize}

\emph{Think about how does your pacakge help people to understand
optimisation}

\begin{itemize}
\tightlist
\item
  diagnostic on stochastic optim
\item
  vis the prograssion of multi-parameter decision variable
\item
  understanding learning rate - neightbourhood parameter
\item
  understand where the local \& global maximum is found - trace plot -
  see if noisy function
\end{itemize}

\hypertarget{recording-the-guided-tour}{%
\section{Recording the guided tour}\label{recording-the-guided-tour}}

\hypertarget{tour-components}{%
\subsection{Tour components}\label{tour-components}}

Guided tour, along with other types of tour, has been implemented in the
\emph{tourr} package in R, available on the Comprehensive R Archive
Network at \url{https://cran.r-project.org/web/packages/tourr/}
\citep{wickham2011tourrpackage}. A tour includes two major components: a
\emph{generator} that generating the projection basis according to
projection pursuit and an \emph{interpolator} that performing geodesic
interpolation between the projection basis.

The psudo-code below illustrates the implementation of guided tour in
the tourr package. Given an projection pursuit index function and a
randomly generated projection basis (current basis), the optimisation
procedure produces a target basis inside \texttt{generator()} Both the
current basis and the target basis will be supplied to
\texttt{tour\_path()} to prepare information needed for constructing a
geodesic path. This information is then used to compute a series of
interpolating bases inside the \texttt{tour()} function. All the basis
will be sent to create animation for visualising the tour in the
\texttt{animate()} function.

\begin{Shaded}
\begin{Highlighting}[]
\NormalTok{animation <-}\StringTok{ }\ControlFlowTok{function}\NormalTok{()\{}
  
  \CommentTok{# compute projection basis }
\NormalTok{  tour <-}\StringTok{ }\ControlFlowTok{function}\NormalTok{()\{}
    
    \CommentTok{# construct bases on the tour path}
\NormalTok{    new_geodesic_path <-}\StringTok{ }\ControlFlowTok{function}\NormalTok{()\{}
\NormalTok{      tour_path <-}\StringTok{ }\ControlFlowTok{function}\NormalTok{()\{}
        
        \CommentTok{# GENERATOR: generate projection basis via projection pursuit}
\NormalTok{        guided_tour <-}\StringTok{ }\ControlFlowTok{function}\NormalTok{()\{}
\NormalTok{          generator <-}\StringTok{ }\ControlFlowTok{function}\NormalTok{()\{}
            
            \CommentTok{# define projection pursuit index}
            \CommentTok{# generate the target basis from the current basis via optimisation}
\NormalTok{          \}}
\NormalTok{        \}}
        
        \CommentTok{# prepare geodesic information needed for interpolating along the tour path}
\NormalTok{      \}}
\NormalTok{    \}}
    
    \CommentTok{# INTERPOLATOR: interpolate between the current and target basis }
    \ControlFlowTok{function}\NormalTok{()\{}
      \CommentTok{# generate interpolating bases on the geodesic path}
\NormalTok{    \}}
\NormalTok{  \}}
  
  \CommentTok{# animate according to different display methods}
\NormalTok{\}}
\end{Highlighting}
\end{Shaded}

\hypertarget{data-structure}{%
\subsection{Data Structure}\label{data-structure}}

In the current implementation of the \texttt{tourr} package, while the
target basis generated by the projection pursuit can be accessed later
via \texttt{save\_history()}, interpolating bases and those randonly
nearby bases generated in the optimisation are not stored. This creates
difficulties for fully understand the behaviour of the optimisation and
interpolation of tour in complex scenario \textbf{{[}needa rephrase this
part{]}.}

Two set of simulated data are used in the demonstration of the
visualiation and diagnostics of the tour optimisation. A small dataset
consists of 1000 randomly simulated observations of five variables
(\texttt{x1}, \texttt{x2}, \texttt{x8}, \texttt{x9}, \texttt{x10}).
\texttt{x2} is the informative variable simulated from two bi-modal
normal distribution centered at -3 and 3 with variance being 1 and the
other four are simulated from \texttt{N(0,\ 1)}. The data has been
scaled to ensure \texttt{x2} has variance of 1.

A larger dataset contains more informative variables (\texttt{x3} to
\texttt{x7}) of different types. \texttt{x3} takes 500 positive one and
500 negative one. The distribution of all the variables except
\texttt{x3} is plotted below. \emph{{[}should I introduce the dist for
each var?{]}}

\includegraphics{paper_files/figure-latex/unnamed-chunk-3-1.pdf}

Once the dataset is sent to the tourr package, all the information
generated will be stored in a global structure. The global structure
consists of six columnes: \texttt{basis}, \texttt{index\_val},
\texttt{tries}, \texttt{info}, \texttt{loop}, \texttt{id} and captured
all the basis generated during whole tour process. The example below
presents the global object of a 1D projection of the small dataset with
geodesic seraching method.

\begin{Shaded}
\begin{Highlighting}[]
\NormalTok{holes_1d_geo }\OperatorTok\StringTok{ }\KeywordTok{head}\NormalTok{(}\DecValTok{5}\NormalTok{)}
\end{Highlighting}
\end{Shaded}

\begin{verbatim}
## # A tibble: 5 x 7
##   basis             index_val tries info              loop method             id
##   <list>                <dbl> <dbl> <chr>            <dbl> <chr>           <int>
## 1 <dbl[,1] [5 x 1]>     0.749     1 start               NA <NA>                1
## 2 <dbl[,1] [5 x 1]>     0.749     1 direction_search     1 search_geodesic     2
## 3 <dbl[,1] [5 x 1]>     0.749     1 direction_search     1 search_geodesic     3
## 4 <dbl[,1] [5 x 1]>     0.749     1 direction_search     1 search_geodesic     4
## 5 <dbl[,1] [5 x 1]>     0.749     1 direction_search     1 search_geodesic     5
\end{verbatim}

\texttt{tries} has an increment of one once the generator is called
(equivalently a new target basis is generated); \texttt{info} records
the stage the basis is in. This would include the \texttt{interpolation}
stage and the detailed stage in the optimisation i.e.
\texttt{direction\_search}, \texttt{best\_direction\_search},
\texttt{line\_search}and \texttt{best\_line\_search} for geodesic
searching (\texttt{search\_geodesic}); \texttt{random\_search} and
\texttt{new\_basis} for simulating annealing (\texttt{search\_better}).
\texttt{loop} is the counter used for the optimisation procedure and
thus will be \texttt{NA} for interpolation steps. \texttt{id} creates a
sequential order of the basis. This information will be stored and
printed when the optimisation ends and can be turned off via
\texttt{print\ =\ FALSE}. Additional messages during the optimisation
can be displayed via \texttt{verbose\ =\ TRUE}. Another examples is a 2D
projection of the larger dataset with two informative variable
(\texttt{x2} and \texttt{x7}) using search\_better method. Notice in
this example, the dimension of the bases becomes 6 by 2.

\begin{Shaded}
\begin{Highlighting}[]
\NormalTok{holes_2d_better }\OperatorTok\StringTok{ }\KeywordTok{head}\NormalTok{(}\DecValTok{5}\NormalTok{) }
\end{Highlighting}
\end{Shaded}

\begin{verbatim}
## # A tibble: 5 x 7
##   basis             index_val tries info           loop method           id
##   <list>                <dbl> <dbl> <chr>         <dbl> <chr>         <int>
## 1 <dbl[,2] [6 x 2]>     0.804     1 start            NA <NA>              1
## 2 <dbl[,2] [6 x 2]>     0.793     1 random_search     1 search_better     2
## 3 <dbl[,2] [6 x 2]>     0.784     1 random_search     2 search_better     3
## 4 <dbl[,2] [6 x 2]>     0.773     1 random_search     3 search_better     4
## 5 <dbl[,2] [6 x 2]>     0.795     1 random_search     4 search_better     5
\end{verbatim}

\hypertarget{visual-methods}{%
\section{Visual methods}\label{visual-methods}}

\hypertarget{add-something-about-general-vis}{%
\subsection{add something about general
vis}\label{add-something-about-general-vis}}

The creation of the global object facilitates the diagnostics of the
optimisation in the projection pursuit guided tourr. Different
diagnostic plots could be made to understand the optimisation
\textbf{{[}pretty cheesy here{]}}

\emph{The examples and results below still need to be modified as the
development / modification of both ferrn and tourr package}.

\hypertarget{index-vs-time-plots}{%
\subsection{index vs time plots}\label{index-vs-time-plots}}

The tracing plot is useful to explore the index value. Given a global
object created, \texttt{explore\_trace\_all()} plots the index value of
all the bases.

\begin{Shaded}
\begin{Highlighting}[]
\NormalTok{holes_2d_geo }\OperatorTok\StringTok{ }\KeywordTok{explore_trace_all}\NormalTok{(}\DataTypeTok{magnify =} \OtherTok{TRUE}\NormalTok{)}
\end{Highlighting}
\end{Shaded}

\includegraphics{paper_files/figure-latex/unnamed-chunk-7-1.pdf}

\begin{Shaded}
\begin{Highlighting}[]
\NormalTok{holes_2d_geo }\OperatorTok\StringTok{ }\KeywordTok{explore_trace_interp}\NormalTok{()}
\end{Highlighting}
\end{Shaded}

\includegraphics{paper_files/figure-latex/unnamed-chunk-7-2.pdf}

\hypertarget{target-basis-vs-interpolation-steps-index-vs-time}{%
\subsection{target basis vs interpolation steps, index vs
time}\label{target-basis-vs-interpolation-steps-index-vs-time}}

\hypertarget{low-d-representation-pca}{%
\subsection{low-d representation, PCA}\label{low-d-representation-pca}}

\begin{Shaded}
\begin{Highlighting}[]
\NormalTok{holes_1d_geo }\OperatorTok\StringTok{ }\KeywordTok{explore_proj_pca}\NormalTok{()}
\end{Highlighting}
\end{Shaded}

\begin{verbatim}
## Warning: The `x` argument of `as_tibble.matrix()` must have column names if `.name_repair` is omitted as of tibble 2.0.0.
## Using compatibility `.name_repair`.
## This warning is displayed once every 8 hours.
## Call `lifecycle::last_warnings()` to see where this warning was generated.
\end{verbatim}

\includegraphics{paper_files/figure-latex/unnamed-chunk-8-1.pdf}

\hypertarget{animating-plots}{%
\section{Animating plots}\label{animating-plots}}

\begin{Shaded}
\begin{Highlighting}[]
\NormalTok{holes_1d_geo }\OperatorTok\StringTok{ }\KeywordTok{explore_proj_pca}\NormalTok{(}\DataTypeTok{animate =} \OtherTok{TRUE}\NormalTok{)}
\end{Highlighting}
\end{Shaded}

\hypertarget{finding-errors-and-developing-improvements}{%
\section{Finding errors and developing
improvements}\label{finding-errors-and-developing-improvements}}

The visualisation methods introduced above allow us to assess the
performance of each optimisation routine and thus provide improvments.

\hypertarget{interrupt}{%
\subsection{Interrupt}\label{interrupt}}

The left panel of Figure @ref(fig:interruption) presents the trace plot
of the interpolated basis using \texttt{search\_better} to optimise the
holes index (2D projection using the larger dataset). Tour interpolates
from the current basis to the target basis, which has been found to have
a higher index value by projection pursuit. The target basis will then
be sent to the projection pursuit as the current basis to find the next
target basis. We can observe from the plot that there are basis with
even higher index value on the interpolation path. These bases could be
used to search for new basis in the next round. Therefore, an
interruption is constructed to only take interpolated basis up to the
point where the index value stops increasing and the last basis is taken
as the current basis for the next round of search. After implementing
the interruption, the tracing plot with the same configuration is shown
on the right panel. Rather than interpolate to the target basis, the
interruption interrupt the process after id = 60 and proceeds the next
round of search from the interpolated basis. Taking advantage of the
interpolated basis results in a higher index value in the end with fewer
steps.

\begin{figure}
\centering
\includegraphics{paper_files/figure-latex/interruption-1.pdf}
\caption{\label{interruption} Trace plots of the interpolated basis with
and without the interruption. The interruption stops the interpolation
when the index value starts to decrease, see the difference at id being
around 60. The implementation of the interuption finds an ending basis
with higher index value using fewer steps.}
\end{figure}

Further the case when the index function is not smooth, the
interpolation may not attain -\textgreater{} modification

\hypertarget{polish}{%
\subsection{Polish}\label{polish}}

In principle, all the optimisation routines should present the same
output on the same problem. In Figure @ref(fig:trace-compare), red dots
shows the trace of the interpolated basis using
\texttt{search\_geodesic} and \texttt{search\_better} (with max.tries =
100), respectively, on the 2D projection problem. We can observe that
they attain slightly different ending index value, which is not ideal.
This motivates the creation of a polishing search that polishes the
ending basis and achieves unity on different methods.

\includegraphics{paper_files/figure-latex/unnamed-chunk-10-1.pdf}

\texttt{search\_polish} takes the ending basis of a given search as the
starting basis and uses a brutal-force approach to sample a larger
number of basis (\texttt{n\_sample}) in the neightbourhood, controlled
by \texttt{polish\_alpha}. Among the \texttt{n\_sample} basis, the one
with the largest index value becomes the candidate. If its index value
is larger than that of the starting basis, it becomes the center of the
searching neightbourhood in the next round. If no basis is found to have
larger index value than the starting basis, the searching neighbourhood
will shrink and the search continues. The polishing search ends when one
of the four stopping criteria is satisfied:

\begin{enumerate}
\def\labelenumi{\arabic{enumi})}
\tightlist
\item
  the two basis can't be too close
\item
  the improvement can't be too small
\item
  the searching neighbourhood can't be too small
\item
  the number of iteration can't exceed the max.tries
\end{enumerate}

The usage of search\_polish is as follows. After the first tour, the
final basis from the interpolation is extracted and supplied into a new
tour with the \texttt{start} argument and \texttt{search\_polish} as the
searching function in the guided\_tour. All the other arguments should
remain the same.

\begin{Shaded}
\begin{Highlighting}[]
\KeywordTok{set.seed}\NormalTok{(}\DecValTok{123456}\NormalTok{)}
\NormalTok{holes_2d_geo <-}\StringTok{ }\KeywordTok{animate_xy}\NormalTok{(data_mult[,}\KeywordTok{c}\NormalTok{(}\DecValTok{1}\NormalTok{,}\DecValTok{2}\NormalTok{, }\DecValTok{7}\OperatorTok{:}\DecValTok{10}\NormalTok{)],}\DataTypeTok{tour_path =} 
                             \KeywordTok{guided_tour}\NormalTok{(}\KeywordTok{holes}\NormalTok{(), }\DataTypeTok{d =} \DecValTok{2}\NormalTok{, }
                                         \DataTypeTok{search_f =}\NormalTok{ tourr}\OperatorTok{:::}\NormalTok{search_geodesic),}
                           \DataTypeTok{rescale =} \OtherTok{FALSE}\NormalTok{, }\DataTypeTok{verbose =} \OtherTok{TRUE}\NormalTok{)}

\NormalTok{last_basis <-}\StringTok{ }\NormalTok{holes_2d_geo }\OperatorTok\StringTok{ }\KeywordTok{filter}\NormalTok{(info }\OperatorTok{==}\StringTok{ "interpolation"}\NormalTok{) }\OperatorTok\StringTok{ }
\StringTok{  }\KeywordTok{tail}\NormalTok{(}\DecValTok{1}\NormalTok{) }\OperatorTok\StringTok{ }\KeywordTok{pull}\NormalTok{(basis) }\OperatorTok\StringTok{ }\NormalTok{.[[}\DecValTok{1}\NormalTok{]]}

\KeywordTok{set.seed}\NormalTok{(}\DecValTok{123456}\NormalTok{)}
\NormalTok{holes_2d_geo_polish <-}\StringTok{ }\KeywordTok{animate_xy}\NormalTok{(data_mult[,}\KeywordTok{c}\NormalTok{(}\DecValTok{1}\NormalTok{,}\DecValTok{2}\NormalTok{, }\DecValTok{7}\OperatorTok{:}\DecValTok{10}\NormalTok{)], }\DataTypeTok{tour_path =} 
                                    \KeywordTok{guided_tour}\NormalTok{(}\KeywordTok{holes}\NormalTok{(), }\DataTypeTok{d =} \DecValTok{2}\NormalTok{, }
                                                \DataTypeTok{search_f =}\NormalTok{ tourr}\OperatorTok{:::}\NormalTok{search_polish),}
                                  \DataTypeTok{rescale =} \OtherTok{FALSE}\NormalTok{, }\DataTypeTok{verbose =} \OtherTok{TRUE}\NormalTok{, }
                                  \DataTypeTok{start =}\NormalTok{ last_basis)}
\end{Highlighting}
\end{Shaded}

The following example conducted a 2D projection on the larger dataset
using search better with different configurations. \texttt{max.tries} is
a hyperparameter that controls the maximum number of try without
improvement and its default value is 25. As shown in Figure
@ref(fig:trace-compare), after polishing, both trials attain the same
index value. However, a small \texttt{max.tries} of 25 is not sufficient
for the algorithm to find the true maximum. This is because 25 tries is
not sufficient for the 2D searching space.

\begin{figure}
\centering
\includegraphics{paper_files/figure-latex/trace-compare-1.pdf}
\caption{Breakdown of index value when using different max.tries in
search better in conjunction with search polish. Both attain the same
final index value after the polishing while using a max.tries 25 is not
sufficient to find the ture maximum.}
\end{figure}

\hypertarget{posse}{%
\subsection{posse?}\label{posse}}

search\_better annd search\_geodesic methods are not doing good on 2d
projection -\textgreater{} see the simple case above: abs best is 0.96
while they only get to 0.88-ish. Would be worse for higher dimension.
thus search\_posse. get 0.95 and then polish to 0.96 :)

\bibliographystyle{agsm}
\bibliography{biblio.bib}

\end{document}
