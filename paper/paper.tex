% !TeX program = pdfLaTeX
\documentclass[12pt]{article}
\usepackage{amsmath}
\usepackage{graphicx,psfrag,epsf}
\usepackage{enumerate}
\usepackage{natbib}
\usepackage{textcomp}
\usepackage[hyphens]{url} % not crucial - just used below for the URL
\usepackage{hyperref}
\providecommand{\tightlist}{%
  \setlength{\itemsep}{0pt}\setlength{\parskip}{0pt}}

%\pdfminorversion=4
% NOTE: To produce blinded version, replace "0" with "1" below.
\newcommand{\blind}{0}

% DON'T change margins - should be 1 inch all around.
\addtolength{\oddsidemargin}{-.5in}%
\addtolength{\evensidemargin}{-.5in}%
\addtolength{\textwidth}{1in}%
\addtolength{\textheight}{1.3in}%
\addtolength{\topmargin}{-.8in}%

%% load any required packages here


\usepackage{color}
\usepackage{fancyvrb}
\newcommand{\VerbBar}{|}
\newcommand{\VERB}{\Verb[commandchars=\\\{\}]}
\DefineVerbatimEnvironment{Highlighting}{Verbatim}{commandchars=\\\{\}}
% Add ',fontsize=\small' for more characters per line
\usepackage{framed}
\definecolor{shadecolor}{RGB}{248,248,248}
\newenvironment{Shaded}{\begin{snugshade}}{\end{snugshade}}
\newcommand{\AlertTok}[1]{\textcolor[rgb]{0.94,0.16,0.16}{#1}}
\newcommand{\AnnotationTok}[1]{\textcolor[rgb]{0.56,0.35,0.01}{\textbf{\textit{#1}}}}
\newcommand{\AttributeTok}[1]{\textcolor[rgb]{0.77,0.63,0.00}{#1}}
\newcommand{\BaseNTok}[1]{\textcolor[rgb]{0.00,0.00,0.81}{#1}}
\newcommand{\BuiltInTok}[1]{#1}
\newcommand{\CharTok}[1]{\textcolor[rgb]{0.31,0.60,0.02}{#1}}
\newcommand{\CommentTok}[1]{\textcolor[rgb]{0.56,0.35,0.01}{\textit{#1}}}
\newcommand{\CommentVarTok}[1]{\textcolor[rgb]{0.56,0.35,0.01}{\textbf{\textit{#1}}}}
\newcommand{\ConstantTok}[1]{\textcolor[rgb]{0.00,0.00,0.00}{#1}}
\newcommand{\ControlFlowTok}[1]{\textcolor[rgb]{0.13,0.29,0.53}{\textbf{#1}}}
\newcommand{\DataTypeTok}[1]{\textcolor[rgb]{0.13,0.29,0.53}{#1}}
\newcommand{\DecValTok}[1]{\textcolor[rgb]{0.00,0.00,0.81}{#1}}
\newcommand{\DocumentationTok}[1]{\textcolor[rgb]{0.56,0.35,0.01}{\textbf{\textit{#1}}}}
\newcommand{\ErrorTok}[1]{\textcolor[rgb]{0.64,0.00,0.00}{\textbf{#1}}}
\newcommand{\ExtensionTok}[1]{#1}
\newcommand{\FloatTok}[1]{\textcolor[rgb]{0.00,0.00,0.81}{#1}}
\newcommand{\FunctionTok}[1]{\textcolor[rgb]{0.00,0.00,0.00}{#1}}
\newcommand{\ImportTok}[1]{#1}
\newcommand{\InformationTok}[1]{\textcolor[rgb]{0.56,0.35,0.01}{\textbf{\textit{#1}}}}
\newcommand{\KeywordTok}[1]{\textcolor[rgb]{0.13,0.29,0.53}{\textbf{#1}}}
\newcommand{\NormalTok}[1]{#1}
\newcommand{\OperatorTok}[1]{\textcolor[rgb]{0.81,0.36,0.00}{\textbf{#1}}}
\newcommand{\OtherTok}[1]{\textcolor[rgb]{0.56,0.35,0.01}{#1}}
\newcommand{\PreprocessorTok}[1]{\textcolor[rgb]{0.56,0.35,0.01}{\textit{#1}}}
\newcommand{\RegionMarkerTok}[1]{#1}
\newcommand{\SpecialCharTok}[1]{\textcolor[rgb]{0.00,0.00,0.00}{#1}}
\newcommand{\SpecialStringTok}[1]{\textcolor[rgb]{0.31,0.60,0.02}{#1}}
\newcommand{\StringTok}[1]{\textcolor[rgb]{0.31,0.60,0.02}{#1}}
\newcommand{\VariableTok}[1]{\textcolor[rgb]{0.00,0.00,0.00}{#1}}
\newcommand{\VerbatimStringTok}[1]{\textcolor[rgb]{0.31,0.60,0.02}{#1}}
\newcommand{\WarningTok}[1]{\textcolor[rgb]{0.56,0.35,0.01}{\textbf{\textit{#1}}}}



\begin{document}


\def\spacingset#1{\renewcommand{\baselinestretch}%
{#1}\small\normalsize} \spacingset{1}


%%%%%%%%%%%%%%%%%%%%%%%%%%%%%%%%%%%%%%%%%%%%%%%%%%%%%%%%%%%%%%%%%%%%%%%%%%%%%%

\if0\blind
{
  \title{\bf Title here}

  \author{
        Author 1 \thanks{The authors gratefully acknowledge \ldots{}} \\
    Department of YYY, University of XXX\\
     and \\     Author 2 \\
    Department of ZZZ, University of WWW\\
      }
  \maketitle
} \fi

\if1\blind
{
  \bigskip
  \bigskip
  \bigskip
  \begin{center}
    {\LARGE\bf Title here}
  \end{center}
  \medskip
} \fi

\bigskip
\begin{abstract}
\begin{enumerate}
\def\labelenumi{\arabic{enumi}.}
\tightlist
\item
  check consistency of using PP for projection pursuit and PPI for
  projection pursuit index
\end{enumerate}
\end{abstract}

\noindent%
{\it Keywords:} 3 to 6 keywords, that do not appear in the title
\vfill

\newpage
\spacingset{1.45} % DON'T change the spacing!

\hypertarget{introduction}{%
\section{Introduction}\label{introduction}}

\hypertarget{tour}{%
\subsection{Tour}\label{tour}}

Tour provides a way to explore multivariate data interactively via an
established tour path. A tour path is formed by interpolating between
randomly generated plane. Different types of tours are available depends
on the purpose of the exploration i.e.~a grand tour is suitable for
randomly exploring the data from different angles; a guided tour detects
a particular structure in the data. a manual tour allows user to
manually control the projection \citep{cook1997manual}.

\hypertarget{guided-tour}{%
\subsection{Guided tour}\label{guided-tour}}

Guided tour is usually used in conjunction projection pursuit, a method
coined by \citet{friedman1974projection} to detect ``interesting''
low-diemsion projection of multivariate data. A projection pursuit
requires the definition of a projection pursuit index function and an
optimisation routine. The projection pursuit index measures the
``interestingness'' of data defined as its departure from normality.
Numerous indices have been proposed in the literature, including
lengendre index \citep{friedman1974projection}, hermite index
\citep{hall1989polynomial}, natural hermite index
\citep{cook1993projection}, chi-square index
\citep{posse1995projection}, LDA index \citep{lee2005projection} and PDA
index \citep{lee2010projection}. An optimisation routine is required to
find the projection basis (thus projection) that maximises the
projection pursuit index. The discussion of existing optimisation
procedure will be discussed in the next section.

Guided tour creates visualisation for the projections found by
projection pursuit by constructing a tour path. An illustration modified
from \citep{buja2005computational} entails the geodesic interpolation
bewteen the plane generated by projection pursuit.

\includegraphics[width=7.75in]{/Users/hzha400/Documents/3.PhD/paper-tour-vis/figures/tour_path_new}

\hypertarget{optimisation-methods-in-projection-pursuit-literature}{%
\subsection{Optimisation methods in projection pursuit
literature}\label{optimisation-methods-in-projection-pursuit-literature}}

As \citet{friedman1974projection} said ``\ldots{}, the technique use for
maximizing the (one- and two-dimensional) projection index strongly
influences both the statistical and the com- putational aspects of the
procedure.'' The quality of the optimisation procedure largely affect
the tour view and thus, the interesting projection one could possibly
observe. An ideal optimisation procedure needs to have following
characteristics:

\begin{itemize}
\item
  \emph{Being able to handle non-differentiable index function}: the
  index function could be noisy and non-differentible
\item
  \emph{Being able to optimise with constraints}: the projection matrix
  is restricted to an orthornormal matrix.
\item
  \emph{Being able to reveal both local and global maximum}: Although
  the primary interest is to find the global maximum, distinct local
  structures are also of our interest.
\end{itemize}

Below present three existing methods in tour.

\citet{posse1995projection} presented a random search algorithm that
samples new basis in the neighbourhood of the current basis. The
neighbourhood is defined via the radius of the p-dimenionsal sphere,
\(c\). The new basis is taken as the target basis if it has higher index
value, or the sampling continues. If no basis is found to have higher
index value after a certain humber of tries \(n\), the radius \(c\) is
halved. The algorithm stops when the maximum numbere of iteration is
attained or the radius \(c\) is less than a pre-determined number.
{[}Pursuit package uses this method and it works greate! But I don't
think we implement this - although don't think it is too hard to do
it{]}.

\citet{cook1995grand} explained the use of a gradient ascent
optimisation with the assumption that the index function is continuous
and differentiable. Since some indices could be non-differentiable, the
computation of derivative is replaced by a psudo-derivative of
evaluating five randomly generated directions in a tiny nearby
neighbourhood. Taking a step on the straight derivative direction has
been modified to maximise the projection pursuit index along the
geodesic direction.

Simulated annealing
\citep[\citet{kirkpatrick1983optimization}]{bertsimas1993simulated} is a
non-derivative procedure based on a non-increasing cooling scheme
\(T(i)\). Given an initial \(T_0\), the temperature at iteration \(i\)
is defined as \(T(i) = \frac{T_0}{log(i + 1)}\). The simulated annealing
algorithm works as follows. Given a neighbourhood parameter \(\alpha\)
and a randomly generated orthonormal basis \(B\), a candidate basis is
constructed as \(B_j = (1- \alpha)B_i + \alpha B\) where \(B_i\) is the
current basis. If the index value of the candidate basis is larger than
the one of the current basis, the candidate basis becomes the target
basis. If it is smaller, the candidate is accepted with probability
\(A = \min \left(\exp(-\frac{I(B_j) - I(B_i)}{T(i)}), 1 \right)\) where
\(I(.)\) is the index function.

\hypertarget{problems-and-difficulties-in-pp-optimisation}{%
\subsection{problems and difficulties in PP
optimisation}\label{problems-and-difficulties-in-pp-optimisation}}

Below listed several issues in projection pursuit optimisation. Some are
general optimisation problems, while others are more specific for PP
optimisation.

\begin{itemize}
\item
  \emph{Finding global maximum}: Although finding local maximum is
  relatively easy with developed algorithms, it is generally hard to
  guarantee global maximum in a problem where the objective function is
  complex or the number of decision variables is large. Also, there are
  discussions on how to avoid getting trapped in a local optimal in the
  literature.
\item
  \emph{optimising non-smooth function}: When the objective function is
  non-differentiable, derivative information can not be obtained, which
  means traditional gradient- or Hessian- based methods are not
  feasible. Stochastic optimisation method could be an alternative to
  solve these problems.
\item
  \emph{computation speed}: The optimisation procedure needs to be fast
  to compute since tours produces real-time animation of the projected
  data.
\item
  \emph{consistency result in stochastic optimisation}: In stochastic
  algorithm, researchers usually set a seed to ensure the algorithm
  producing the same result for every run. This practice supports
  reproducibility, while less efforts has been made to guarantee
  different seeds will provide the same result.
\item
  \emph{high-dimensional decision variable}: In projection pursuit, the
  decision variable includes all the entries in the projection matrix,
  which is high-dimensional. Researcher would be better off if they can
  understand the relative position of different projection matrix in the
  high-dimensional space.
\item
  \emph{role of interpolation in PP optimisation}: An optimisation
  procedure usually involves iteratively finding projection bases that
  maximises the index function, while tour requires geodesic
  interpolation between these bases to produce a continuous view for the
  users. It would be interesting to see if the interpolated bases could,
  in reverse, help the optimisation reach faster convergence.
\end{itemize}

\emph{Think about how does your pacakge help people to understand
optimisation}

\begin{itemize}
\tightlist
\item
  diagnostic on stochastic optim
\item
  vis the prograssion of multi-parameter decision variable
\item
  understanding learning rate - neightbourhood parameter
\item
  understand where the local \& global maximum is found - trace plot -
  see if noisy function
\end{itemize}

\hypertarget{iterative-algorithm-and-its-diagnostics}{%
\section{Iterative algorithm and its
diagnostics}\label{iterative-algorithm-and-its-diagnostics}}

\hypertarget{tour-components}{%
\subsection{Tour components}\label{tour-components}}

Guided tour, along with other types of tour, has been implemented in the
\emph{tourr} package in R, available on the Comprehensive R Archive
Network at \url{https://cran.r-project.org/web/packages/tourr/}
\citep{wickham2011tourrpackage}. A tour includes two major components: a
\emph{generator} that generating the projection basis according to
projection pursuit and an \emph{interpolator} that performing geodesic
interpolation between the projection basis.

The psudo-code below illustrates the implementation of guided tour in
the tourr package. Given an projection pursuit index function and a
randomly generated projection basis (current basis), the optimisation
procedure produces a target basis inside \texttt{generator()} Both the
current basis and the target basis will be supplied to
\texttt{tour\_path()} to prepare information needed for constructing a
geodesic path. This information is then used to compute a series of
interpolating bases inside the \texttt{tour()} function. All the basis
will be sent to create animation for visualising the tour in the
\texttt{animate()} function.

\begin{Shaded}
\begin{Highlighting}[]
\NormalTok{animation <-}\StringTok{ }\ControlFlowTok{function}\NormalTok{()\{}
  
  \CommentTok{# compute projection basis }
\NormalTok{  tour <-}\StringTok{ }\ControlFlowTok{function}\NormalTok{()\{}
    
    \CommentTok{# construct bases on the tour path}
\NormalTok{    new_geodesic_path <-}\StringTok{ }\ControlFlowTok{function}\NormalTok{()\{}
\NormalTok{      tour_path <-}\StringTok{ }\ControlFlowTok{function}\NormalTok{()\{}
        
        \CommentTok{# GENERATOR: generate projection basis via projection pursuit}
\NormalTok{        guided_tour <-}\StringTok{ }\ControlFlowTok{function}\NormalTok{()\{}
\NormalTok{          generator <-}\StringTok{ }\ControlFlowTok{function}\NormalTok{()\{}
            
            \CommentTok{# define projection pursuit index}
            \CommentTok{# generate the target basis from the current basis via optimisation}
\NormalTok{          \}}
\NormalTok{        \}}
        
        \CommentTok{# prepare geodesic information needed for interpolating along the tour path}
\NormalTok{      \}}
\NormalTok{    \}}
    
    \CommentTok{# INTERPOLATOR: interpolate between the current and target basis }
    \ControlFlowTok{function}\NormalTok{()\{}
      \CommentTok{# generate interpolating bases on the geodesic path}
\NormalTok{    \}}
\NormalTok{  \}}
  
  \CommentTok{# animate according to different display methods}
\NormalTok{\}}
\end{Highlighting}
\end{Shaded}

\hypertarget{diagnostics}{%
\subsection{Diagnostics}\label{diagnostics}}

Visualisatioin has been widely used for exploring and understanding
data. Visualisation presents information in a graphicial manner and
often allows people to see information they would otherwise not seen in
the reporting of numerical summarisation. Visual diagnostics can be
real-time or post-run. Real time diagnostic directly uses the data
produced in the algorithm to produce visual representation and thus
doesn't need to store the data. This section focuses on the definition
and production of post-run diagnoistics and the next section discusses
real-time diagnostics.

Post-run diagnostics requires the data produced during the algorithm to
be stored/ saved in order to produce plot diagnostics.

The graphical system in R is established based on the conception gramma
of graphic \citep{wickham2010layered}, where a graphic is defined using
stacked layers in a coordination system. A layer includes 1) the dataset
that powers the plot; 2) a geometric object that represents the visual
shape of the plot and 3) relevant statistical transformation if needed.
An important concept in the gramma of graphic is \emph{aesthetic
mapping}. Aesthetic mapping links the variable in a dataset to
information needed to produce a geometric object. For example, we map
one variable on the x-axis and another on the y-axis to create a
scatterplot. To create a boxplot, we first map one variable on the
x-axis and then map the five point summary of another variable on the
y-axis. This computation of five point summary from the origin variable
is statistical transformation. This definition of graphic through layers
provides advantages to produce complex plots since all the plots,
however complex, can be decomposed into basic geometric objects shaped
by the variable supplied.

This idea of decompositing a graph to basic elements inspires me to
characterise the diagnostics of iterative algorithm in a similar
fashion. The optimisation in many machine learning algorithms these days
is iterative in nature, while remains as a black box. Being able to
diagnose it visually allows researchers to have the tool to unfold the
myth and thus provides more understand to the algorithm.

The concept of gramma of graphic requires a dataset to be supplied in
making a plot, thus a global object needs to be created in the iterative
algorithm. The variables will be mapped into element of a graphic to
explore and thus should contain all the parameters of interest. The next
section shows how a global object is created in projection pursuit
guided tour.

\hypertarget{data-structure}{%
\subsection{Data Structure}\label{data-structure}}

In the current implementation of the \texttt{tourr} package, while the
target basis generated by the projection pursuit can be accessed later
via \texttt{save\_history()}, interpolating bases and those randonly
nearby bases generated in the optimisation are not stored. This creates
difficulties for fully understand the behaviour of the optimisation and
interpolation of tour in complex scenario \textbf{{[}needa rephrase this
part{]}.}

Two set of simulated data are used in the demonstration of the
visualiation and diagnostics of the tour optimisation. A small dataset
consists of 1000 randomly simulated observations of five variables
(\texttt{x1}, \texttt{x2}, \texttt{x8}, \texttt{x9}, \texttt{x10}).
\texttt{x2} is the informative variable simulated from two bi-modal
normal distribution centered at -3 and 3 with variance being 1 and the
other four are simulated from \texttt{N(0,\ 1)}. The data has been
scaled to ensure \texttt{x2} has variance of 1.

A larger dataset contains more informative variables (\texttt{x3} to
\texttt{x7}) of different types. \texttt{x3} takes 500 positive one and
500 negative one. The distribution of all the variables except
\texttt{x3} is plotted below. \emph{{[}should I introduce the dist for
each var?{]}}

\includegraphics{paper_files/figure-latex/origin-data-1.pdf}

Once the dataset is sent to the tourr package, all the information
generated will be stored in a global structure. The global structure
consists of six columnes: \texttt{basis}, \texttt{index\_val},
\texttt{tries}, \texttt{info}, \texttt{loop}, \texttt{id} and captured
all the basis generated during whole tour process. The example below
presents the global object of a 1D projection of the small dataset with
geodesic seraching method.

\begin{Shaded}
\begin{Highlighting}[]
\NormalTok{holes_1d_geo }\OperatorTok\StringTok{ }\KeywordTok{head}\NormalTok{(}\DecValTok{5}\NormalTok{)}
\end{Highlighting}
\end{Shaded}

\begin{verbatim}
## # A tibble: 5 x 8
##   basis           index_val tries info            loop method        alpha    id
##   <list>              <dbl> <dbl> <chr>          <dbl> <chr>         <dbl> <int>
## 1 <dbl[,1] [5 x ~     0.749     1 start             NA <NA>            0.5     1
## 2 <dbl[,1] [5 x ~     0.749     1 direction_sea~     1 search_geode~  NA       2
## 3 <dbl[,1] [5 x ~     0.749     1 direction_sea~     1 search_geode~  NA       3
## 4 <dbl[,1] [5 x ~     0.749     1 direction_sea~     1 search_geode~  NA       4
## 5 <dbl[,1] [5 x ~     0.749     1 direction_sea~     1 search_geode~  NA       5
\end{verbatim}

\texttt{tries} has an increment of one once the generator is called
(equivalently a new target basis is generated); \texttt{info} records
the stage the basis is in. This would include the \texttt{interpolation}
stage and the detailed stage in the optimisation i.e.
\texttt{direction\_search}, \texttt{best\_direction\_search},
\texttt{line\_search}and \texttt{best\_line\_search} for geodesic
searching (\texttt{search\_geodesic}); \texttt{random\_search} and
\texttt{new\_basis} for simulating annealing (\texttt{search\_better}).
\texttt{loop} is the counter used for the optimisation procedure and
thus will be \texttt{NA} for interpolation steps. \texttt{id} creates a
sequential order of the basis. This information will be stored and
printed when the optimisation ends and can be turned off via
\texttt{print\ =\ FALSE}. Additional messages during the optimisation
can be displayed via \texttt{verbose\ =\ TRUE}. Another examples is a 2D
projection of the larger dataset with two informative variable
(\texttt{x2} and \texttt{x7}) using search\_better method. Notice in
this example, the dimension of the bases becomes 6 by 2.

\begin{Shaded}
\begin{Highlighting}[]
\NormalTok{holes_2d_better }\OperatorTok\StringTok{ }\KeywordTok{head}\NormalTok{(}\DecValTok{5}\NormalTok{) }
\end{Highlighting}
\end{Shaded}

\begin{verbatim}
## # A tibble: 5 x 8
##   basis             index_val tries info           loop method       alpha    id
##   <list>                <dbl> <dbl> <chr>         <dbl> <chr>        <dbl> <int>
## 1 <dbl[,2] [6 x 2]>     0.804     1 start            NA <NA>           0.5     1
## 2 <dbl[,2] [6 x 2]>     0.793     1 random_search     1 search_bett~   0.5     2
## 3 <dbl[,2] [6 x 2]>     0.784     1 random_search     2 search_bett~   0.5     3
## 4 <dbl[,2] [6 x 2]>     0.773     1 random_search     3 search_bett~   0.5     4
## 5 <dbl[,2] [6 x 2]>     0.795     1 random_search     4 search_bett~   0.5     5
\end{verbatim}

\hypertarget{post-run-static-diagnostics-plots}{%
\section{Post-run static diagnostics
plots}\label{post-run-static-diagnostics-plots}}

Given all the information of interest in a global object, the next thing
is to design its visual representation that can diagnose our algorithm.
In an iterative algorithm, one would be interested to know how one
parameter changes as the algorithm iterates. The parameter could be
could be the decision variable, the value of the optimisation function
or other parameter that facilitate the algorithm. In projectio pursuit
guided tour, apart from the decision variable projection basis, the
value of index function (\texttt{index\_val}), we are also interested to
explore the effect the different searching methods, the neighbourhood
parameter alpha, and the stage a particular observation is in
(\texttt{info}). These are the variables that will be mapped into the
diagnostic plot. The global object also has its time series nature since
the observations are recorded as the algorithm progresses and thus can't
be switch in rows. This time series structure is always hierachical and
nested since each row can be labelled with the smallest unit \texttt{id}
and up by the iterative structure of the algorithm. Take our example, in
the global object, each row has an \texttt{id} label which is the
smallest unit the observation is ordered. A larger ordering unit is
\texttt{tries} since it is updated everytime a new target basis is
found. A nested ordering unit is \texttt{loop}, which increases by one
as searching method iterates and starts over at a new \texttt{tries}.
These features explain the definition of the global object for
projection pursuit guided tour in Figure \ref{fig:glb-obj}.

\begin{figure}
\includegraphics[width=1\linewidth,height=0.2\textheight]{/Users/hzha400/Documents/3.PhD/paper-tour-vis/figures/global_obj} \caption{\label{glb-obj}this is xxxx}\label{fig:glb-obj}
\end{figure}

\hypertarget{explore-scalar-parameters}{%
\subsection{Explore scalar parameters}\label{explore-scalar-parameters}}

The most interesting parameter to explore is the value of objective
function. The points recorded in the global object can be divided into
two broad categories: searching points and interpolating points

\begin{itemize}
\item
  \emph{Searching points} include the observations that are recorded in
  the searching algorithm in order to find the target basis. The points
  for target bases is also included in the searching points and there is
  one such point per \texttt{tries}.
\item
  \emph{interpolating points} exist in the guided tour to produce
  continuous animated view from one target basis to another and it
  doesn't have \texttt{loop} value.
\end{itemize}

\hypertarget{explore-searching-points}{%
\subsubsection{Explore searching
points}\label{explore-searching-points}}

As mentioned previously, the largest difficulties of exploring searching
points is its unknown number of observations per \texttt{tries}. Mapping
\texttt{id} on the x-axis will leave the \texttt{tries} with few
observations a small space in the plot, while those \texttt{tries} with
large number of search points towards the end occupying the vast
majority of the space in the plot.

This motivates the use of summarised statistics. At each iteration,
rather than knowing the index value of \emph{every} points, we are more
interested to know a general summary of all the points and more
importantly, the point with the largest \texttt{index\_val} since it
prescribes the geodesic interpolation and future searches.

Boxplot is a suitable candidate that provides five points summary of the
data, while it has one drawback: it doesn't report the number of point
in each box. We may risk losing information on how long it takes for the
search to find the target basis by displaying the boxplot alone for all
\texttt{tries}. Thus, the number of point for each \texttt{tries} is
displayed at the bottom of each box and we provide options to switch
\texttt{tries} with small number of points to a point geometry. This is
achieved via the \texttt{cutoff} argument. A line geometry is also added
to link the points with the largest index value for each \texttt{tries}.
This helps to visualise the improvement made by each \texttt{tries}.

\emph{Example: exploring searching points} The data is sourced from a 2D
projection of the larger dataset and \texttt{search\_better} is used
with \texttt{max.tries\ =\ 500}. In Figure \ref{fig:points-id} and
\ref{fig:points-tries}, a comparison has been made on visualising the
searching points with \texttt{id} and \texttt{tries} on the x-axis,
colored by \texttt{tries}. In Figure \ref{fig:points-id}, the searching
points of the first few \texttt{tries} are squeezed in a small width
leaving the large uninteresting searching points in the last three
\texttt{tries} taking a vast majority of the plotting space. While in
Figure \ref{fig:points-tries} the data is spaced by \texttt{tries}
evenly in the plot. Label at the bottom indicates the number of
observations in each tries and facilitates the choice of cutoff to
switch from point geometry to boxplot geometry (\texttt{cutoff\ =\ 15}).
The line geometry suggest the largest improvement happens at
\texttt{tries\ =\ 5}.

\begin{figure}
\centering
\includegraphics{paper_files/figure-latex/points-id-1.pdf}
\caption{\label{points-id}this is a xxx}
\end{figure}

\begin{figure}
\centering
\includegraphics{paper_files/figure-latex/points-tries-1.pdf}
\caption{\label{points}A comparison of plotting the same search points
with different plot designs. The left plot doesn't efficiently use the
plot space to convey information from the plot while the right plot
provides good summarisation of data and number of points in each tries.}
\end{figure}

\hypertarget{explore-interpolating-points}{%
\subsubsection{Explore interpolating
points}\label{explore-interpolating-points}}

Plotting the interpolating points as time series data allows us to
diagnose characterstics of different configurations and index functions.
Here we present two examples of using plots to diagnose the tour
algorithm and different index functions.

\emph{Example: Interruption} This examples uses \texttt{search\_better}
for a 2D projection on the larger dataset using the \texttt{holes}
index. As mentioned previously, the interpolation starts from the
current basis to the target basis, which has been found by the
projection pursuit algorithm to ahve a higher index value. After the
interpolation, the target basis will become the current basis and send
back to the projection pursuit algorithm to find the next target basis.
From figure \ref{fig:interruption}, it is possible that there are bases
with index value higher than the target basis on the interpolation path
and these bases could be used to search for new basis in the next
iteration.

Thus an interruption is constructed to accpet the interpolating bases up
to the one with the larget index value on the interpolation path, and
that basis is taken as the current basis for the next iteration. After
implementing this interruption, the tracing plot with the same
configuration is shown on the lower panel. We can observe that rather
than interpolating to the target basis at \texttt{id\ =\ 62}, the
interpolation stops before the index value starts to decrease at
\texttt{id\ =\ 60}. This implementation results in a higher index value
in the end with fewer steps.

\begin{figure}
\centering
\includegraphics{paper_files/figure-latex/interruption-1.pdf}
\caption{\label{interruption}Trace plots of the interpolated basis with
and without the interruption. The interruption stops the interpolation
when the index value starts to decrease at id = 60. The implementation
of the interuption finds an ending basis with higher index value using
fewer steps.}
\end{figure}

\emph{Example: Noisy index function} Add an example on noisy function

xxx

Currently, I have three things to put for the noisy index function.

\begin{enumerate}
\def\labelenumi{\arabic{enumi})}
\item
  interpolating plot finds that search\_geodesic is not doing a good job
  on optimising the noisy index; while search\_better is.
\item
  The interpolating plot for noisy index function is not smooth but the
  target basis is always increasing.
\item
  search\_better\_random sees to find both local and global maximum.
\end{enumerate}

Need to decise where to put this example. Do we wan to intoduce noisy
function at this stage or we put it here because they are insights draw
from the interpolating plots.

xxx

\hypertarget{adding-more-variables-to-the-mapping}{%
\subsubsection{Adding more variables to the
mapping}\label{adding-more-variables-to-the-mapping}}

At previous two sections, only the iterator and the index value are
mapped onto the x and y aesthetics of the plot; while more aesthetics
i.e.~color, could be added to compare other parameters in the global
object. Two examples are shown below to explore and compare different
searching methods and neighbourhood parameter alpha.

\emph{Example: Polish} In principle, all the optimisation routines
should result in the same output on the same problem. Figure
\ref{fig:trace-compare} shows the interpolating plots for two different
searching methods: search\_geodesic and serach\_better on a 2D
projection problem for the larger dataset using the holes index. The two
methods reaches different ending value in the red dots, which is not
ideal. This motivates the creation of a polishing search that polishes
the ending basis and achieves unity on different methods.

\includegraphics{paper_files/figure-latex/trace-compare-1.pdf}

\texttt{search\_polish} takes the ending basis of a given search as the
current basis and uses a brutal-force approach to sample a larg number
of basis (\texttt{n\_sample}) in the neightbourhood, whose radius is
controlled by \texttt{polish\_alpha}. Among the \texttt{n\_sample}
basis, the one with the largest index value becomes the candidate. If
its index value of the candidate basis is larger than that of the
current basis, it becomes the current basis in the next iteration. If no
basis is found to have larger index value than the current basis, the
searching neighbourhood will be shrunk and the search continues. The
polishing search ends when one of the four stopping criteria is
satisfied:

\begin{enumerate}
\def\labelenumi{\arabic{enumi})}
\tightlist
\item
  the two basis can't be too close
\item
  the percentage improvement of the index function can't be too small
\item
  the searching neighbourhood can't be too small
\item
  the number of iteration can't exceed the \texttt{max.tries}
\end{enumerate}

The usage of search\_polish is as follows. After the first tour, the
final basis from the interpolation is extracted and supplied into a new
tour with the \texttt{start} argument and \texttt{search\_polish} as the
searching function in the guided\_tour. All the other arguments should
remain the same.

\begin{Shaded}
\begin{Highlighting}[]
\KeywordTok{set.seed}\NormalTok{(}\DecValTok{123456}\NormalTok{)}
\NormalTok{holes_2d_geo <-}\StringTok{ }\KeywordTok{animate_xy}\NormalTok{(data_mult[,}\KeywordTok{c}\NormalTok{(}\DecValTok{1}\NormalTok{,}\DecValTok{2}\NormalTok{, }\DecValTok{7}\OperatorTok{:}\DecValTok{10}\NormalTok{)],}\DataTypeTok{tour_path =} 
                             \KeywordTok{guided_tour}\NormalTok{(}\KeywordTok{holes}\NormalTok{(), }\DataTypeTok{d =} \DecValTok{2}\NormalTok{, }
                                         \DataTypeTok{search_f =}\NormalTok{ tourr}\OperatorTok{:::}\NormalTok{search_geodesic),}
                           \DataTypeTok{rescale =} \OtherTok{FALSE}\NormalTok{, }\DataTypeTok{verbose =} \OtherTok{TRUE}\NormalTok{)}

\NormalTok{last_basis <-}\StringTok{ }\NormalTok{holes_2d_geo }\OperatorTok\StringTok{ }\KeywordTok{filter}\NormalTok{(info }\OperatorTok{==}\StringTok{ "interpolation"}\NormalTok{) }\OperatorTok\StringTok{ }
\StringTok{  }\KeywordTok{tail}\NormalTok{(}\DecValTok{1}\NormalTok{) }\OperatorTok\StringTok{ }\KeywordTok{pull}\NormalTok{(basis) }\OperatorTok\StringTok{ }\NormalTok{.[[}\DecValTok{1}\NormalTok{]]}

\KeywordTok{set.seed}\NormalTok{(}\DecValTok{123456}\NormalTok{)}
\NormalTok{holes_2d_geo_polish <-}\StringTok{ }\KeywordTok{animate_xy}\NormalTok{(data_mult[,}\KeywordTok{c}\NormalTok{(}\DecValTok{1}\NormalTok{,}\DecValTok{2}\NormalTok{, }\DecValTok{7}\OperatorTok{:}\DecValTok{10}\NormalTok{)], }\DataTypeTok{tour_path =} 
                                    \KeywordTok{guided_tour}\NormalTok{(}\KeywordTok{holes}\NormalTok{(), }\DataTypeTok{d =} \DecValTok{2}\NormalTok{, }
                                                \DataTypeTok{search_f =}\NormalTok{ tourr}\OperatorTok{:::}\NormalTok{search_polish),}
                                  \DataTypeTok{rescale =} \OtherTok{FALSE}\NormalTok{, }\DataTypeTok{verbose =} \OtherTok{TRUE}\NormalTok{, }
                                  \DataTypeTok{start =}\NormalTok{ last_basis)}
\end{Highlighting}
\end{Shaded}

The following example conducted a 2D projection on the larger dataset
using search better with different configurations. \texttt{max.tries} is
a hyperparameter that controls the maximum number of try without
improvement and its default value is 25. As shown in Figure
\ref{fig:trace-compare}, after polishing, both trials attain the same
index value. However, a small \texttt{max.tries} of 25 is not sufficient
for the algorithm to find the true maximum. This is because 25 tries is
not sufficient for the 2D searching space.

\begin{figure}
\centering
\includegraphics{paper_files/figure-latex/polish-1.pdf}
\caption{\label{trace-compare}Breakdown of index value when using
different max.tries in search better in conjunction with search polish.
Both attain the same final index value after the polishing while using a
max.tries 25 is not sufficient to find the ture maximum.}
\end{figure}

\begin{Shaded}
\begin{Highlighting}[]
\CommentTok{# nrow(holes_2d_better_max_tries)}
\CommentTok{# nrow(holes_2d_pos)}
\end{Highlighting}
\end{Shaded}

\hypertarget{explore-matrix-parameter}{%
\subsection{Explore matrix parameter}\label{explore-matrix-parameter}}

Matrix parameter could also be interested to explore, for example, in
the tour, we are interested to explore the relative position of the
projection basis in the vector space. A projection basis is a
vector/matrix whose row number is the number of variable in the given
dataset. This imposes difficulties in visualisation since we are bounded
to preceive at most three dimensions. Thus, principal component analysis
is used to reduce the dimension of the projection basi and the first two
principal component are mapped to the x and y axis of the plot. Another
variable of interest could be mapped to the color aesthetics to see how
it changes as the projection basis changes in the projected 2D space.
{[}have a second read here{]}

\emph{Example: understand search\_geodesic via mapping \texttt{info} to
color} \texttt{search\_geodesic} is a two-stage ascending algorithm with
four different stages in the search and a PCA plot useful to understand
how the algorithm works. Starting from the start basis, a directional
search is conducted in a narrow neighbourhood on five random directions.
The best one is picked and a line search is then run on the geodesic
direction to find the target basis. The starting and target basis are
then interpolated. In the next iteration, the target basis becomes the
current basis and then procedures continues. {[}should probably reword
this part with info levels xxx{]}

\begin{figure}
\centering
\includegraphics{paper_files/figure-latex/pca-1.pdf}
\caption{\label{pca}this is xxx}
\end{figure}

\emph{Example: understand the polishing alpha via mapping alpha to
color} \texttt{search\_polish} is a brutal-force algorithm that evaluate
1000 points in the neighbourhod at each loop and an appropriate inital
value in neighbourhood parameter alpha would avoid search on large
vector space and concentrate on the ending basis. In Figure, two
different initial alpha value is used in the polishing step. {[}some
more illustration of the comparison of the two polish alpha here{]}

\hypertarget{real-time-animated-diagnostic-plots}{%
\section{Real-time animated diagnostic
plots}\label{real-time-animated-diagnostic-plots}}

\begin{Shaded}
\begin{Highlighting}[]
\NormalTok{holes_1d_geo }\OperatorTok\StringTok{ }\KeywordTok{explore_proj_pca}\NormalTok{(}\DataTypeTok{animate =} \OtherTok{TRUE}\NormalTok{, }\DataTypeTok{col =}\NormalTok{ info)}
\end{Highlighting}
\end{Shaded}

\hypertarget{vis-package}{%
\section{Vis package}\label{vis-package}}

Everything is coded up in a package.

\clearpage

\bibliographystyle{agsm}
\bibliography{biblio.bib}

\end{document}
